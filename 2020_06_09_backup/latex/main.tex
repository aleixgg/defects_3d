\documentclass[letterpaper]{article}
\usepackage[utf8]{inputenc}
\usepackage{amssymb,amsmath,amsfonts}
\usepackage{amsthm}
\usepackage[
  pdfusetitle,
  bookmarksnumbered,
  hidelinks,
  colorlinks,
  citecolor=blue,
  linktoc=page
]{hyperref}
\usepackage[backgroundcolor=white,linecolor=green!60]{todonotes}
\usepackage{dsfont}
\usepackage{geometry}
\usepackage{graphicx}
\usepackage{caption}
\usepackage{subcaption}
\usepackage{pgfplots}
\pgfplotsset{compat=1.15}

% Keep figures in right section
\usepackage{placeins}
\let\Oldsection\section
\renewcommand{\section}{\FloatBarrier\Oldsection}
\let\Oldsubsection\subsection
\renewcommand{\subsection}{\FloatBarrier\Oldsubsection}
\let\Oldsubsubsection\subsubsection
\renewcommand{\subsubsection}{\FloatBarrier\Oldsubsubsection}


\newcommand\be{\begin{equation}}
\newcommand\bea{\begin{eqnarray}}
\newcommand\eea{\end{eqnarray}}
\newcommand\ee{\end{equation}}

\newcommand\eg{{\it e.g.}}
\newcommand\ie{{\it i.e.}}
\newcommand\cf{{\it cf.}}
\newcommand\mf{\mathfrak}

\def\Am{{\mathcal{A}}}
\def\Bm{{\mathcal{B}}}
\def\Cm{{\mathcal{C}}}
\def\Dm{{\mathcal{D}}}
\def\Em{{\mathcal{E}}}
\def\Fm{{\mathcal{F}}}
\def\Gm{{\mathcal{G}}}
\def\Hm{{\mathcal{H}}}
\def\Im{{\mathcal{I}}}
\def\Km{{\mathcal{K}}}
\def\Mm{{\mathcal{M}}}
\def\Nm{{\mathcal{N}}}
\def\Om{{\mathcal{O}}}
\def\Pm{{\mathcal{P}}}
\def\Qm{{\mathcal{Q}}}
\def\Rm{{\mathcal{R}}}
\def\Sm{{\mathcal{S}}}
\def\Um{{\mathcal{U}}}
\def\Xm{{\mathcal{X}}}

\def\Mb{{\bar{M}}}
\def\Qb{{\bar{Q}}}
\def\Sb{{\bar{S}}}

\def\a{{\alpha}}
\def\b{{\beta}}
\def\c{{\gamma}}
\def\d{{\delta}}
\def\ad{{\dot{\alpha}}}
\def\bd{{\dot{\beta}}}
\def\cd{{\dot{\gamma}}}
\def\dd{{\dot{\delta}}}
\def\jbar{{\bar{\jmath}}}

\def\Xb{{\mathbf{X}}}
\def\eps{\epsilon}
\def\veps{\varepsilon}
\def\ph{\phantom}
\def\pd{\partial}
\def\w{\omega}
\def\wb{\bar{\omega}}
\def\nn{\nonumber}

%%%%%%%%%%%%%%%%%%%%%%%%%%%%%%%%%%%%%%%%%%%%%%%%%%%
%Brackets
%%%%%%%%%%%%%%%%%%%%%%%%%%%%%%%%%%%%%%%%%%%%%%%%%%%
\newcommand{\bra}[1]{\ensuremath{\left< #1\,\right|}}
\newcommand{\ket}[1]{\ensuremath{\left|\, #1\right>}}
\newcommand{\bracket}[2]{\ensuremath{\left< #1\, |\, #2\right>}}
\newcommand{\glbracket}[2]{\ensuremath{\left< #1\, ,\, #2\right>}}
\newcommand{\normalorder}[1]{:\, #1\,:}
\newcommand{\melement}[3]{\ensuremath{\left< #1\left| \, #2\,\right| #3\right>}}
\newcommand{\vev}[1]{\ensuremath{\melement{0}{#1}{0}}}
\newcommand{\vac}[1]{\ensuremath{\left< \, #1\, \right>}}
\newcommand{\expvalue}[2]{\ensuremath{\melement{#2}{#1}{#2}}}
\newcommand{\com}[2]{\ensuremath{\left[ #1\, ,\, #2\right]}}
\newcommand{\acom}[2]{\ensuremath{\left\{ #1\, ,\, #2\right\}}}
\newcommand{\form}[2]{\ensuremath{\left( #1, #2\right)}}
\newcommand{\Nc}{{\ensuremath{\mathcal{N}}}}

%%%%%%%%%%%%%%%%%%%%%%%%%%%%%%%%%%%%%%%%%%%%%%%%%%%
%Algebra Letters
%%%%%%%%%%%%%%%%%%%%%%%%%%%%%%%%%%%%%%%%%%%%%%%%%%%
\newcommand{\fS}{{\ensuremath{\mathsf{S}}}}\newcommand{\fP}{{\ensuremath{\mathsf{P}}}}
\newcommand{\fQ}{{\ensuremath{\mathsf{Q}}}}\newcommand{\fL}{{\ensuremath{\mathsf{L}}}}
\newcommand{\fK}{{\ensuremath{\mathsf{K}}}}\newcommand{\fD}{{\ensuremath{\mathsf{D}}}}
\newcommand{\fq}{{\ensuremath{\mathsf{q}}}}\newcommand{\fr}{{\ensuremath{\mathsf{r}}}}
\newcommand{\fs}{{\ensuremath{\mathsf{s}}}}\newcommand{\ft}{{\ensuremath{\mathsf{t}}}}
\newcommand{\fu}{{\ensuremath{\mathsf{u}}}}

\newcommand\zb{{\bar z}}
\newcommand\dcox{h^{\vee}}

\newcommand\qq{\mathbbmtt{Q}}
\newcommand\q{\mathbbmtt{Q}\,_{1}}
\newcommand\qd{\mathbbmtt{Q}\,_{2}}


\newcommand\uf{{\mf{u}}}
\newcommand\sof{{\mf{so}}}
\newcommand\suf{{\mf{su}}}
\newcommand\ospf{{\mf{osp}}}
\newcommand\uspf{{\mf{usp}}}
\newcommand\slf{{\mf{sl}}}
\newcommand\ef{{\mf{e}}}
\newcommand\gf{{\mf{g}}}
\newcommand\hhf{{\mf{h}}}
\newcommand\Rf{{\mf{R}}}

\newcommand\SO{\mathrm{SO}}
\newcommand\SU{\mathrm{SU}}
\newcommand\USp{\mathrm{USp}}

\def \uf{{\mf{u}}}
\def \sof{{\mf{so}}}
\def \suf{{\mf{su}}}
\def \slf{{\mf{sl}}}
\def \ef{{\mf{e}}}
\def \gf{{\mf{g}}}
\def \hhf{{\mf{h}}}
\def \Rf{{\mf{R}}}


% Lie Algebras
\newcommand{\gl}{\ensuremath{\mathfrak{gl}}}%
\newcommand{\so}{\ensuremath{\mathfrak{so}}}%
\newcommand{\su}{\ensuremath{\mathfrak{su}}}%
\newcommand{\osp}{\ensuremath{\mathfrak{osp}}}%
\renewcommand{\sp}{\ensuremath{\mathfrak{sp}}}%
\renewcommand{\sl}{\ensuremath{\mathfrak{sl}}}%
 
% Other
\def \ph{\phantom}
\DeclareMathOperator{\tr}{tr}
\DeclareMathOperator{\str}{str}
% \DeclareMathOperator{\deg}{deg}
% \newcommand{\dd}{\ensuremath{\mathrm{d}}} % Differential
\newcommand{\px}{\ensuremath{\partial_x}} % Partial derivative w.r.t. x
\newcommand{\pt}[2]{\ensuremath{\partial^{#1}_{#2}}} % Partial derivative w.r.t. theta
\newcommand{\cc}{\ensuremath{\mathfrak c_2}} % Casimir eigenvalue

% Allow to increase spacing in pmatrix
\makeatletter
\renewcommand*\env@matrix[1][\arraystretch]{%
  \edef\arraystretch{#1}%
  \hskip -\arraycolsep
  \let\@ifnextchar\new@ifnextchar
  \array{*\c@MaxMatrixCols c}}
\makeatother

%opening
\title{Defects across dimensions}
\author{}

\begin{document}

\maketitle

\section{Superconformal algebra}

We use the conventions from Bobev and friends \texttt{1503.02081}, which we now summarize.
We work in euclidean signature, so upper and lower indices do not matter.
The conformal algebra is:
\begin{align}
 [ D, P_\mu] & =   P_\mu, \\
 [ D, K_\mu] & = - K_\mu, \\
 [ K_\mu, P_\nu] & = 2 \left( \delta_{\mu\nu} D - M_{\mu\nu} \right), \\
 [ M_{\mu\nu}, M_{\rho\sigma} ] & = -\delta_{\mu\rho} M_{\nu\sigma} \pm \ldots, \\
 [ M_{\mu\nu}, P_\rho] & = - \delta_{\mu\rho} P_\nu + \delta_{\nu\rho} P_\mu, \\
 [ M_{\mu\nu}, K_\rho] & = - \delta_{\mu\rho} K_\nu + \delta_{\nu\rho} K_\mu.
\end{align}
The supercharges transform as:
\begin{align}
 [D, Q^+_\a ] & = \tfrac{1}{2} Q^+_\a, \\
 [D, Q^-_\ad] & = \tfrac{1}{2} Q^-_\ad, \\
 [R, Q^+_\a ] & =  Q^+_\a, \\
 [R, Q^-_\ad] & = -Q^-_\ad, \\
 [K^\mu, Q^+_\a]  & = \Sigma^\mu_{\a\ad} S^{\ad +}, \\
 [K^\mu, Q^-_\ad] & = \Sigma^\mu_{\a\ad} S^{\a  -}, \\
 [M_{\mu\nu}, Q^+_\a]  & = (       m_{\mu\nu})_\a^{\ph \a\b}    Q^{+}_\b, \\
 [M_{\mu\nu}, Q^-_\ad] & = (\tilde m_{\mu\nu})^\bd_{\ph \bd\ad} Q^{-}_\bd,
\end{align}
and
\begin{align}
 [D, S^{\ad+} ] & = - \tfrac{1}{2} S^{\ad+}, \\
 [D, S^{\a -} ] & = - \tfrac{1}{2} S^{\a -}, \\
 [R, S^{\ad+} ] & =  S^{\ad+}, \\
 [R, S^{\a -} ] & = -S^{\a -}, \\
 [P_\mu, S^{\ad+}] & = - \bar \Sigma_\mu^{\ad\a} Q^+_\a, \\
 [P_\mu, S^{\a -}] & = - \bar \Sigma_\mu^{\ad\a} Q^-_\ad, \\
 [M_{\mu\nu}, S^{\ad+}] & = - (\tilde m_{\mu\nu})^\ad_{\ph \ad\bd} S^{\bd+}, \\
 [M_{\mu\nu}, S^{\a -}] & = - (       m_{\mu\nu})_\b ^{\ph \b\a}   S^{\b-}.
\end{align}
The anticommutators are:
\begin{align}
 \{ Q^+_\a, Q^-_\ad \}   & =      \Sigma^\mu_{\a\ad} P_\mu, \\
 \{ S^{\ad+}, S^{\a-} \} & = \bar \Sigma^{\ad\a}_\mu K_\mu, \\
 \{ S^{\ad+}, Q^-_\bd \} 
   & = \delta^\ad_{\ph \ad\bd} \left(D + \frac{d-1}{2} R \right) 
     -  (\tilde m_{\mu\nu})^\ad_{\ph \ad\bd} M_{\mu\nu}
     + \ldots, \\
 \{ S^{\a-}, Q^+_\b \} 
   & = \delta^\a_{\ph \a\b} \left(D - \frac{d-1}{2} R \right) 
     - (m_{\mu\nu})_\b^{\ph \b\a} M_{\mu\nu}
     + \ldots.
\end{align}
The gamma matrices are defined from
\begin{align}
 & \Sigma_i \bar \Sigma_j + \Sigma_j \bar \Sigma_i = 2 \delta_{ij}, \\
 & \bar \Sigma_i \Sigma_j + \bar \Sigma_j \Sigma_i = 2 \delta_{ij}, \\
 & \bar \Sigma^{\ad\a}_\mu = \left( \Sigma^\mu_{\a\ad} \right)^*,
\end{align}
and then
\begin{align}
 m_{\mu\nu} & = \frac{1}{4} \left( 
    \Sigma_\nu \bar \Sigma_\mu - 
    \Sigma_\mu \bar \Sigma_\nu
 \right), \\
 \tilde m_{\mu\nu} & = \frac{1}{4} \left( 
    \bar \Sigma_\mu \Sigma_\nu - 
    \bar \Sigma_\nu \Sigma_\mu
 \right).
\end{align}

The Casimir is
\begin{align}
 C_2 = 
    D^2
  - \frac12 \{ P_\mu, K^\mu \}
  - \frac12 M_{\mu\nu} M^{\mu\nu}
  - \frac12 R^2
  + \frac12 [ S^{\ad+}, Q^-_\ad]
  + \frac12 [ S^{\a-}, Q^+_\a].
\end{align}
Acting with it on an operator with quantum numbers $\Delta, \ell, r$ gives the eigenvalue
\begin{align}
 \lambda_C 
 = \Delta (\Delta - d + 2)
 + \ell(\ell + d - 2)
 - \frac{r^2}{2}.
\end{align}

When we look at the particular case $d = 3$ we take indices $\mu,\nu = 1, 2, 3$ and build the gamma matrices from the Pauli matrices
\begin{align}
 \Sigma^1 = \sigma^1, \quad
 \Sigma^2 = \sigma^2, \quad
 \Sigma^3 = \sigma^3,
\end{align}
When we look at the particular case $d = 4$ we take indices $\mu,\nu = 1, 2, 3, 4$ and build the remaining matrix as
\begin{align}
 \Sigma^4 = i \mathds{1}_{2}.
\end{align}

\subsection{Action of generators on fields}

We will consider two-point functions $\langle \phi_1(x_1) \phi_2(x_2) \rangle$, where $\phi_1$ is chiral and $\phi_2$ is antichiral
\begin{align}
 & [Q^+_\a,  \phi_1(0)] = 0, \\
 & [Q^-_\ad, \phi_2(0)] = 0.
\end{align}
Consistency with the superconformal algebra fixes the $R$-charge in terms of the dimension:
\begin{align}
 & [D, \phi_1(0)] = \Delta_\phi, \\
 & [D, \phi_2(0)] = \Delta_\phi, \\
 & [R, \phi_1(0)] = \frac{2}{d-1} \Delta_\phi, \\
 & [R, \phi_2(0)] = \frac{2}{d-1} \Delta_\phi.
\end{align}
The operators are translated as
\begin{align}
 \phi(x) = e^{x\cdot P} \phi(0) e^{-x\cdot P},
\end{align}
which implies the usual
\begin{align}
 [P_\mu, \phi(x)] & = \partial_\mu \phi(x), \\
 [D, \phi(x)] & =  (x^\mu \partial_\mu + \Delta) \phi(x), \\
 [M_{\mu\nu}, \phi(x)] & =  (x_\nu \partial_\mu - x_\mu \partial_\nu) \phi(x), \\
 \ldots
\end{align}
The shortening condition implies
\begin{align}
 [Q^+_\a,   \phi_1(x_1)] & = 0, \\
 [S^{\ad+}, \phi_1(x_1)] & = 0, \\
 [Q^-_\ad,  \phi_2(x_2)] & = 0, \\
 [S^{\a-},  \phi_2(x_2)] & = 0.
\end{align}
For the remaining $S$ generators we get
\begin{align}
 & [S^{\a-}, \phi_1(x_1)]
 = x_1^\mu \bar \Sigma_\mu^{\ad\a} [Q^-_\ad, \phi_1(x_1)], \\
 & [S^{\ad+}, \phi_2(x_2)]
 = x_2^\mu \bar \Sigma_\mu^{\ad\a} [Q^+_\a,  \phi_2(x_2)].
\end{align}

\section{Half-BPS boundaries}

\subsection{Non-supersymmetric}

We can obtain a subalgebra restricting to
\begin{align}
 D, \quad
 R, \quad
 M_{ab}, \quad
 P_{a}, \quad
 K_{a},
\end{align}
where $a,b = 1, \ldots, d-1$ are parallel to the boundary, which sits at $x^d = 0$.
The defect Casimir is obtained from the full one restricting to the defect operators
\begin{align}
 C_{\text{def}} = 
    D^2
  - \frac12 \{ P_{a}, K^{a} \}
  - \frac{1}{2} M_{ab} M^{ab}
\end{align}
The Ward identities fix the two-point function
\begin{align}
 \langle \phi_1(x_1) \phi_2(x_2) \rangle
 = \frac{f(\xi)}{(2x_1^d)^{\Delta_1} (2x_2^d)^{\Delta_2}}, \qquad
 \xi = \frac{x_{12}^2}{4 x_1^{d} x_2^{d}}.
\end{align}

\subsubsection{Defect channel}

Acting with the bosonic part of the defect Casimir and using the non-supersymmetric Casimir $\lambda = \Delta(\Delta - d + 1)$ we get in general dimension
\begin{align}
 \frac{
 \left( C_{\text{def,bos}} - \lambda \right) \langle \phi_1(x_1) \phi_2(x_2)\rangle
 }{\langle \phi_1(x_1) \phi_2(x_2)\rangle}
 = \xi  (\xi +1) f''(\xi ) 
 + \frac{d}{2} (2 \xi +1) f'(\xi )
 - \Delta (\Delta - d + 1) f(\xi )
 = 0,
\end{align}
so the defect block that solves this is
\begin{align}
 f_\Delta(\xi) =
 \xi ^{-\Delta } \, _2F_1\left(\Delta ,\Delta-\frac{d}{2} + 1;2 \Delta -d + 2; -\frac{1}{\xi }\right).
\end{align}

\subsubsection{Bulk channel}

We should act with the full Casimir. 
The bosonic differential equation is
\begin{align}
\begin{split}
 \frac{
 \left( C_{\text{bulk,bos}} - \lambda \right) \langle \phi_1(x_1) \phi_2(x_2)\rangle
 }{\langle \phi_1(x_1) \phi_2(x_2)\rangle}
 & = 4 \xi ^2 (\xi +1) f''(\xi )
   + \big( 4 (2 \Delta_\phi+1) \xi  (\xi +1) - 2 d \xi \big) f'(\xi ) \\
 & \quad 
   +  \big(-\Delta(\Delta -d)-2 d \Delta_\phi+4 \Delta_\phi^2 (\xi +1)\big) f(\xi )
   = 0,
\end{split}
\end{align}
where the bosonic eigenvalue is $\lambda = \Delta(\Delta - d)$.
The block is usually written in terms of $G(\xi) = \xi^{\Delta_\phi} f(\xi)$ because the dependance on the external dimensions drops:
\begin{align}
 G_\Delta(\xi)
 = \xi ^{\Delta /2} \, _2F_1\left(
    \frac{\Delta}{2},
    \frac{\Delta}{2};
    \Delta -\frac{d}{2} + 1;
    -\xi 
 \right).
\end{align}


\subsection{\texorpdfstring{$\Nm = (2, 0)$}{N=(2,0)} subalgebra 
for \texorpdfstring{$d = 3$}{d=3}}

We can obtain a subalgebra restricting to
\begin{align}
 D, \quad
 R, \quad
 M_{12}, \quad
 P_1, P_2, \quad
 K_1, K_2, \quad
 Q^+_1, Q^-_2, \quad
 S^{1-}, S^{2+}.
\end{align}
This can be mapped to the $(2,0)$ algebra in $2d$.
The defect Casimir is obtained from the full one restricting to the defect operators
\begin{align}
 C_{\text{def}} = 
    D^2
  - \frac12 \{ P_{a}, K^{a} \}
  - M_{12}^2
  - \frac12 R^2
  + \frac12 [ S^{2+}, Q^-_2]
  + \frac12 [ S^{1-}, Q^+_1].
\end{align}
Acting with it on an operator with quantum numbers $\Delta, \ell, r$ gives the eigenvalue:
\begin{align}
 \lambda_C 
 = \Delta (\Delta - 1)
 + \ell(\ell - 1)
 - \frac{r^2}{2}.
\end{align}


\subsubsection{Defect channel}

We should act with the defect Casimir at position $x_1$. 
The new contribution from susy is 
\begin{align}
 C_{\text{ferm}} =
 - \frac{1}{2} R^2 
 + \frac{1}{2} [S^{2+}, Q^-_2] 
 + \frac{1}{2} [S^{1-}, Q^+_1] 
 = - Q^-_2 S^{2+} 
   + S^{1-} Q^+_1 
   - \frac{1}{2} R^2 
   + R.
\end{align}
Acting on one point and using the chirality of $\phi_1$ we get
\begin{align}
\begin{split}
 [C_{\text{ferm}}, \phi_1(x_1) ] |0\rangle
 & = \left( -\frac{1}{2} \Delta_\phi^2 + \Delta_\phi \right) \phi_1(x_1) |0\rangle.
\end{split}
\end{align}
This suggests that to obtain the supersymmetric block we should just add a term
$( -\frac{1}{2} \Delta_\phi^2 + \Delta_\phi ) f(\xi)$ in the Casimir equation.
Consider a superprimary $\hat \Om_{p}$ of dimension $\Delta$, spin $\ell = 1/2$ and charge $r = r_\phi - 1$, which has Casimir
\begin{align}
 \lambda 
 = \Delta(\Delta-1) + \ell(\ell - 1) - \frac{r^2}{2} 
 = \Delta(\Delta-1) -\frac{1}{2} \Delta_\phi^2 + \Delta_\phi - \frac{3}{4}.
\end{align}
Then only the descendant $Q^+_{1} \hat \Om_{p}$ has the right quantum numbers to appear in the defect OPE $\phi \sim Q^+_{1} \hat \Om_{p}$, so the superconformal block should reduce to a single non-supersymmetric block.
Indeed, in the differential equation all $\Delta_\phi$ terms drop and we get
\begin{align}
 \xi  (\xi +1) f''(\xi )+\frac{3}{2} (2 \xi +1) f'(\xi )
 -\left(\Delta + \frac{1}{2} \right) \left( \Delta + \frac{1}{2} - 2 \right) f(\xi ) = 0,
\end{align}
which is solved by a shifted non-supersymmetric block $f_{\Delta+\frac{1}{2}}(\xi)$ as expected.


\subsubsection{Bulk channel}

We should act with the full Casimir.
The only new contributions from susy are
\begin{align}
 C_{\text{ferm}} =
 \frac{1}{2} [S^{\ad+}, Q^-_\ad] + 
 \frac{1}{2} [S^{\a-}, Q^+_\a] =
 -Q^-_\ad S^{\ad+} + S^{\ad-} Q^+_\a + 2R.
\end{align}
There is a contribution $-\frac12 R^2$ that cancels between the two sides of the Casimir equation.
Acting on the two points we get
\begin{align}
\begin{split}
 [C_{\text{ferm}}, & \phi_1(x_1) \phi_2(x_2) ] |0\rangle \\
 & =
 \left( 
    [S^{\a-},  \phi_1(x_1)] [Q^+_\a,   \phi_2(x_2)]
  - [Q^-_\ad,  \phi_1(x_1)] [S^{\ad+}, \phi_2(x_2)]
  + 4 \Delta_\phi \, \phi_1(x_1) \phi_2(x_2)
 \right) |0\rangle \\
 & = 
 \left(
    x_{12}^\mu \bar \Sigma_\mu^{\ad\a} [Q^-_\ad, \phi_1(x_1)] [Q^+_\a, \phi_2(x_2)]
  + 4 \Delta_\phi \, \phi_1(x_1) \phi_2(x_2)
 \right) |0\rangle
\end{split}
\end{align}
We use Ward identities to figure them out.
We are only allowed to use $Q^+_1$, $Q^-_2$, $S^{1-}$, $S^{2+}$ to get Ward identities, because they are the only supercharges preserved by our defect.
The first Ward identity
\begin{align}
 \langle \{ Q_1^+, [Q_\ad^-, \phi_1(x_1)] \phi_2(x_2) \} \rangle = 0
\end{align}
implies
\begin{align}
 \langle [Q_\ad^-, \phi_1(x_1)] [Q_1^+, \phi_2(x_2)] \rangle 
 & = \langle [ \{Q_1^+, Q_\ad^-\}, \phi_1(x_1)] \phi_2(x_2) \rangle \\
 & = \Sigma^\mu_{1\ad} \partial_\mu^{x_1} \langle \phi_1(x_1) \phi_2(x_2) \rangle. 
\end{align}
% Explicitly
% \begin{align}
%  \langle [Q_1^-, \phi_1(x_1)] [Q_1^+, \phi_2(x_2)] \rangle 
%  & = -i \partial_2^{x_1} \langle \phi_1(x_1) \phi_2(x_2) \rangle \\
%  \langle [Q_2^-, \phi_1(x_1)] [Q_1^+, \phi_2(x_2)] \rangle 
%  & = \left( -i \partial_0^{x_1} - \partial_1^{x_1} \right) 
%      \langle \phi_1(x_1) \phi_2(x_2) \rangle.
% \end{align}
We can play the same game with the other $Q^\pm$ and $S^\pm$ preserved by the defect.
% The second Ward identity
% \begin{align}
%  \langle \{ Q_2^-, \phi_1(x_1) [Q_2^+, \phi_2(x_2)] \} \rangle = 0
% \end{align}
% implies
% \begin{align}
%  \langle [Q_2^-, \phi_1(x_1)] [Q_2^+, \phi_2(x_2)] \rangle 
%  = - i \partial_2^{x_2} \langle \phi_1(x_1) \phi_2(x_2) \rangle.
% \end{align}
% The third Ward identity
% \begin{align}
%  \langle [ S^{2+}, (Q_1^- \phi_1(x_1)) \phi_2(x_2) ] \rangle = 0
% \end{align}
% is a bit more complicated, and it implies
% \begin{align}
%  \langle [Q_1^-, \phi_1(x_1)] [Q_2^+, \phi_2(x_2)] \rangle 
%  = \frac{1}{x_2^2} \left[ 
%  \left( i x_{12}^0 - x_{12}^1 \right) \partial_2^{x_1}
%  - x_1^2 \left( i \partial_0^{x_1} - \partial_1^{x_1} \right)
%  \right] \langle \phi_1(x_1) \phi_2(x_2) \rangle.
% \end{align}
% All in all we get (we use notation $x^\mu_i$ where $i$ labels point and $\mu=0,1,2$)
% \begin{align}
%  - x_1^2 \partial_2^{x_2}
%  - x_2^2 \partial_2^{x_1}
%  - \frac{1}{x_1^2} \left[
%       ((x_{12}^0)^2 + (x_{12}^1)^2) \partial_2^{x_2}
%     + x_{12}^0 \partial_0^{x_2} + x_{12}^1 \partial_1^{x_2}
%     + i (x_{12}^0 \partial_1^{x_2} - x_{12}^1 \partial_0^{x_2})
%  \right]
%  + 4 \Delta_\phi
% \end{align}
All in all, when we act with the fermionic part of the Casimir on the two-point function we get
\begin{align}
 % 4 \xi  (\xi +1) f'(\xi )+2 \Delta_\phi (2 \xi +1) f(\xi )
 \frac{C_{\text{bulk,ferm}}\langle \phi_1(x_1) \phi_2(x_2)\rangle}
      {\langle \phi_1(x_1) \phi_2(x_2)\rangle}
 = 
 4 \xi  (\xi +1) f'(\xi )+4 \Delta_\phi (\xi +1) f(\xi ).
\end{align}
It is fairly non-trivial to get a differential operator that acting on the two-point function can be rewritten in terms of $\xi$.
This is a sanity check for our calculation.
The supersymmetric Casimir eigenvalue is $\lambda = \Delta(\Delta-1) = \lambda_{bos} + 2\Delta$, so we should also add a term $-2 \Delta f(\xi)$ to the LHS.
Adding all the contributions we get
\begin{align}
\begin{split}
    4 \xi ^2 (\xi +1) f''(\xi )
& + 2 \xi  (4 \Delta_\phi (\xi +1) + 4 \xi +1) f'(\xi ) \\
& + \left(
     2 \Delta_\phi (2 \Delta_\phi (\xi +1)+2 \xi -1) -\Delta(\Delta-1)
   \right) f(\xi ) 
 = 0.
\end{split}
\end{align}
If we define $G(\xi) = \xi^{\Delta_\phi} f(\xi)$ the dependence on $\Delta_\phi$ drops
\begin{align}
   4 \xi ^2 (\xi +1)  G''(\xi )
 + 2 \xi (4 \xi +1)   G' (\xi )
 - \Delta (\Delta -1) G  (\xi ) = 0,
\end{align}
and we find the solution
\begin{align}
 G(\xi) = \xi ^{\Delta /2} \, _2F_1\left(\frac{\Delta }{2}+1,\frac{\Delta }{2};\Delta +\frac{1}{2};-\xi \right),
\end{align}
which is exactly what Philine already found!



\subsection{\texorpdfstring{$\Nm = (1, 1)$}{N=(1,1)} subalgebra
for \texorpdfstring{$d = 3$}{d=3}}

We can obtain a subalgebra restricting to
\begin{align}
 D, \quad
 M_{12}, \quad
 P_1, P_2, \quad
 K_1, K_2, \quad
 Q^+_1 + Q^-_2, Q^+_2 + Q^-_1, \quad
 S^{1+} + S^{2-}, S^{2+} + S^{1-}.
\end{align}
This can be mapped to the $(1,1)$ algebra in $2d$.
The defect Casimir now changes a bit
\begin{align}
 C_{\text{def}} = 
    D^2
  - \frac12 \{ P_{a}, K^{a} \}
  - M_{12}^2
  + \frac14 [ S^{2+} + S^{1-}, Q^+_1 + Q^-_2]
  + \frac14 [ S^{1+} + S^{2-}, Q^+_2 + Q^-_1].
\end{align}
Acting with it on an operator with quantum numbers $\Delta, \ell$ gives the eigenvalue:
\begin{align}
 \lambda_C 
 = \Delta (\Delta - 1)
 + \ell^2.
\end{align}


\subsubsection{Defect channel}

The fermionic contribution to the Casimir equation is
\begin{align}
    \frac14 [ S^{2+} + S^{1-}, Q^+_1 + Q^-_2]
  + \frac14 [ S^{1+} + S^{2-}, Q^+_2 + Q^-_1]
  \to R - \frac{1}{2} \left( 
    Q_2^- S^{1-} + Q_1^- S^{2-}
  \right) + \ldots,
\end{align}
where we dropped all terms that vanish when acting on $\phi_1(x_1)$:
\begin{align}
 [C_{\text{def,ferm}}, \phi_1(x_1)] |0\rangle 
 = \left( \Delta_\phi \phi_1(x_1) + x_1^3 \{ Q_1^-, [Q_2^-, \phi_1(x_1)]\} \right)|0\rangle.
\end{align}
Using Ward identities we express $Q_1^- Q_2^-$ as a differential operator acting at $x_1$, and we get
\begin{align}
 \frac{C_{\text{bulk,ferm}} \langle \phi_1(x_1) \phi_2(x_2) \rangle}
      {\langle \phi_1(x_1) \phi_2(x_2) \rangle}
 = -\xi  f'(\xi).
\end{align}
We also need a contribution $-\Delta f(\xi)$ from the difference in the Casimir eigenvalue, so
\begin{align}
   \xi  (\xi +1) f''(\xi )
 + \left(2 \xi +\frac{3}{2}\right) f'(\xi )
 - \Delta  (\Delta -1)f(\xi )
 = 0.
\end{align}
This is solved by the same block Philine found:
\begin{align}
 f(\xi) 
 = \xi ^{-\Delta } \, _2F_1\left(\Delta ,\frac{1}{2} (2 \Delta -1);2 \Delta ;-\frac{1}{\xi }\right).
\end{align}


\subsubsection{Bulk channel}

The bosonic part and supersymmetric parts are the same as before, the only difference are the Ward identites we can use to rewrite $\langle Q\phi Q\phi \rangle$ in terms of derivatives of $\langle \phi \phi \rangle$.
As before, we can only use the preserved supercharges, so the Ward identities are
\begin{align}
\begin{split}
 \langle \{ Q^+_1 + Q^-_2, [Q^-_2, \phi_1] \phi_2 \} \rangle & = 0, \\
 \langle \{ Q^+_2 + Q^-_1, [Q^-_1, \phi_1] \phi_2 \} \rangle & = 0, \\
 \ldots
\end{split}
\end{align}
% They imply
% \begin{align}
% \begin{split}
%  \langle [Q^-_2, \phi_1] [Q^+_1, \phi_2] \rangle 
%  & = (-i \partial_0^{x_1} - \partial_1^{x_1}) \langle \phi_1 \phi_2 \rangle, \\
%  \langle [Q^-_1, \phi_1] [Q^+_2, \phi_2] \rangle 
%  & = (-i \partial_0^{x_1} + \partial_1^{x_1}) \langle \phi_1 \phi_2 \rangle, \\
% \end{split}
% \end{align}
All in all we get:
\begin{align}
 \frac{C_{\text{bulk,ferm}} \langle \phi_1(x_1) \phi_2(x_2) \rangle}
      {\langle \phi_1(x_1) \phi_2(x_2) \rangle}
 = 4 \xi  f'(\xi )
 + 4 \Delta_\phi f(\xi ).
\end{align}
Following the same arguments as before, we find the blocks
\begin{align}
 G(\xi) 
 = \xi ^{\Delta /2} \, _2F_1\left(\frac{\Delta }{2},\frac{\Delta }{2};\Delta +\frac{1}{2};-\xi \right),
\end{align}
consistent with Philine's results once more!

\subsection{\texorpdfstring{$OSp(1|4)$}{OSP(1|4)} subalgebra
for \texorpdfstring{$d = 4$}{d=4}}

We can obtain a subalgebra restricting to
\begin{align}
 D, \quad
 M_{12}, M_{13}, M_{23}, \quad
 P_1, P_2, P_3, \quad
 K_1, K_2, K_3, \quad
 Q^+_1 + Q^-_2, Q^+_2 - Q^-_1, \quad
 S^{1+} - S^{2-}, S^{2+} + S^{1-}.
\end{align}
This can be mapped to the $OSp(1|4)$ algebra in $3d$.
The defect Casimir now changes a bit
\begin{align}
 C_{\text{def}} = 
    D^2
  - \frac12 \{ P_{a}, K^{a} \}
  - \frac12 M_{ab} M^{ab}
  + \frac14 [ S^{2+} + S^{1-}, Q^+_1 + Q^-_2]
  - \frac14 [ S^{1+} - S^{2-}, Q^+_2 - Q^-_1].
\end{align}
We are using $a, b = 1, 2, 3$ and the boundary sits at $x^4 = 0$.
Acting with it on an operator with quantum numbers $\Delta, \ell$ gives the eigenvalue:
\begin{align}
 \lambda_C 
 = \Delta (\Delta - 2)
 + \ell(\ell + 1).
\end{align}

\subsubsection{Defect channel}

The relevant part of the Casimir is
\begin{align}
 C_{\text{def,ferm}} = - \frac12 Q_2^- S^{1-} + \frac12 Q_1^- S^{2-} + \frac32 R + \ldots,
\end{align}
which acting at point 1 gives
\begin{align}
 [C_{\text{def,ferm}}, \phi_1(x_1)] |0 \rangle
 = \left[ x_1^3 \{ Q_1^-, [Q_2^-, \phi_1(x_1)] \} + \Delta_\phi \phi_1(x_1)
 \right] | 0 \rangle.
\end{align}
Using Ward identities we find
\begin{align}
 \frac{C_{\text{def,ferm}} \langle \phi_1(x_1) \phi_2(x_2) \rangle}
      {\langle \phi_1(x_1) \phi_2(x_2) \rangle}
 = - \xi  f'(\xi ).
\end{align}
This is the same as the $\Nm = (1, 1)$ boundary in $3d$.

\subsubsection{Bulk channel}

It works as before, but the Ward identities are different. The result is
\begin{align}
 \frac{C_{\text{bulk,ferm}} \langle \phi_1(x_1) \phi_2(x_2) \rangle}
      {\langle \phi_1(x_1) \phi_2(x_2) \rangle}
 = 4 \xi  f'(\xi )
 + 4 \Delta_\phi f(\xi ).
\end{align}
This is once again the same as for the $\Nm = (1, 1)$ boundary in $3d$.

\subsection{Boundary across dimensions}

We conjecture that the fermionic part of the Casimir gives the same contribution regardless of the dimension $2 \le d \le 4$. 
(We should still check the boundary in $2d$, but it probably also works there)
Combining the bosonic and supersymmetric parts we get blocks ``across dimensions''.

\subsubsection{Defect channel}

The solution is
\begin{align}
\begin{split}
 f_\Delta^{\text{SUSY}}(\xi) 
 & = \xi ^{-\Delta } \, _2F_1\left(\Delta ,\frac{1}{2} (2 \Delta + 2-d);2 \Delta +3-d;-\frac{1}{\xi }\right) \\
 & = f_\Delta(\xi) + \frac{\Delta }{4\Delta - 2d + 6} f_{\Delta+1}(\xi).
\end{split}
\end{align}


\subsubsection{Bulk channel}

The solution is
\begin{align}
\begin{split}
 G_\Delta^{\text{SUSY}}(\xi) 
 & = \xi ^{\Delta /2} \, _2F_1\left(\frac{\Delta }{2},\frac{\Delta }{2};\Delta +2 -\frac{d}{2};-\xi \right) \\
 & = G_\Delta(\xi) 
   + \frac{\Delta ^2}{(2 \Delta -d +2) (2 \Delta -d +4)} G_{\Delta+2}(\xi).
\end{split}
\end{align}

\section{Codimension two objects}

\subsection{Non-supersymmetric}

We can obtain a subalgebra restricting to
\begin{align}
 D, \quad
 R, \quad
 P_{a}, \quad
 K_{a}, \quad
 M_{ab}, \quad
 M_{ij}.
\end{align}
where $a,b = 1, \ldots, p$ live on the defect and $i,j= p+1, \ldots, d$ are orthogonal. Also $p$ is the dimension of the defect and $q  = d-p$ is the codimension.
The defect Casimir is obtained from the full one restricting to the defect operators.
It factorizes into two commuting pieces
\begin{align}
 C_{\text{def}, 1} = 
    D^2
  - \frac12 \{ P_{a}, K^{a} \}
  - \frac{1}{2} M_{ab} M^{ab}, \qquad
 C_{\text{def}, 2} 
  = - \frac{1}{2} M_{ij} M^{ij}.
\end{align}
The Ward identities fix the two-point function but now there are two cross-ratios
\begin{align}
 \langle \phi_1(x_1) \phi_2(x_2) \rangle
 = \frac{f(\xi, \eta)}{|x_1^\bot|^{\Delta_1} |x_2^\bot|^{\Delta_2}}, \qquad
 \xi = \frac{x_{12}^2}{|x_1^{\bot}| |x_2^{\bot}|}, \qquad
 \eta = \frac{x_{1}^\bot \cdot x_2^\bot}{|x_1^{\bot}| |x_2^{\bot}|}.
\end{align}
For the defect channel we will also use 
\begin{align}
 \chi = \frac{ x_{12}^2 + 2 x_1^\bot \cdot x_2^\bot}
             { |x_1^\bot| |x_2^\bot| } .
\end{align}
In the defect channel, using the normalization of Billo and friends, we get
\begin{align}
 f(\chi, \eta) 
 \to \chi^{-\Delta} \,
 2^{-s}
 \begin{pmatrix}
  s+ \frac{q}{2} -2 \\
  \frac{q}{2} -2
 \end{pmatrix}^{-1}
 C_{s}^{q/2-1}(\eta) 
 \qquad \text{as} \qquad
 \chi \to \infty.
\end{align}



\subsubsection{Defect channel}

The action of the conformal group gives, with $\lambda = \Delta(\Delta-p)$:
\begin{align}
 \frac{(C_{\text{def,1}} - \lambda) \langle \phi_1(x_1) \phi_2(x_2) \rangle}
      {\langle \phi_1(x_1) \phi_2(x_2) \rangle}
 = (\chi^2 - 4) f''(\chi) + (p+1) \chi f'(\chi) - \Delta(\Delta-p) f(\chi)
 = 0.
\end{align}
The action of transverse rotations gives, with $\lambda = s(s + q - 2)$:
\begin{align}
 \frac{(C_{\text{def,2}} - \lambda) \langle \phi_1(x_1) \phi_2(x_2) \rangle}
      {\langle \phi_1(x_1) \phi_2(x_2) \rangle}
 = (\eta^2 - 1) f''(\eta) + (q-1) \eta f'(\eta) - s(s+q-2) f(\eta)
 = 0.
\end{align}
In total the block is
\begin{align}
 \hat f_{\Delta,s}(\chi,\eta) 
 = \alpha_{s,q} \chi ^{-\Delta } 
 \, _2F_1\left(\frac{\Delta }{2},\frac{\Delta +1}{2};\Delta+1-\frac{p}{2};\frac{4}{\chi ^2}\right)
 \, _2F_1\left(-\frac{s}{2},\frac{q+s-2}{2};\frac{q-1}{2};1-\eta ^2\right),
\end{align}
with normalization
\begin{align}
 \alpha_{s,q} = 2^{-s} \frac{\Gamma(q+s-2) \Gamma(q/2-1)}{\Gamma(q/2+s-1) \Gamma(q-2)}.
\end{align}


\subsubsection{Bulk channel}

\subsection{Non-supersymmetric with \texorpdfstring{$U(1)$}{U(1)} mixing}

Let's imagine we have an extra non-compact $U(1)_R$ symmetry with an Hermitian generator $R$.
Then for codimension-two defect, the transverse rotations $SO(2) \simeq U(1)$ can mix with the $U(1)_R$ symmetry
\begin{align}
 M_{12} + i R
\end{align}
(We can always normalize the $R$ generator such that the above equation holds).
The second Casimir is modified to
\begin{align}
C_{\text{def},2} = - (M_{12} + i R)^2.
\end{align}

\subsubsection{Bulk-defect correlator in \texorpdfstring{$3d$}{3d}}

Consider
\begin{align}
 \langle \phi(x_1) \hat \Om(x_2) \rangle
\end{align}
Using $P_3$ it can depend on
\begin{align}
 x_1^1, \quad
 x_1^2, \quad
 x_{12}^3
\end{align}
Using $M_{12}$ we find that $r_1 = r_2 + s_2$ and 
\begin{align}
 (x_1^\perp)^2, \quad
 x_{12}^3
\end{align}
Using dilatations and $K_3$ (and assuming $s_2 = 0$) we find the usual expression
\begin{align}
 \langle \phi(x_1) \hat \Om(x_2) \rangle
 = \frac{1}{[(x_1^\perp)^2 + (x_{12}^3)^2]^{\Delta_2}
 |x_1^\perp|^{\Delta_1 - \Delta_2}} 
\end{align}
Could we have $s_2 \ne 0$?!?

\subsubsection{Defect channel}

Since we assume (from the previous discussion) that the exchanged operator has $r = r_1$ and $s = 0$ the first Casimir equation is the same as before, and the second gives
\begin{align}
 \frac{(C_{\text{def,2}} - \lambda) \langle \phi_1(x_1) \phi_2(x_2) \rangle}
      {\langle \phi_1(x_1) \phi_2(x_2) \rangle}
 = (\eta^2 - 1) f''(\eta) + (\eta + 2r \sqrt{\eta^2-1}) f'(\eta)
 = 0.
\end{align}
The contribution from $R^2$ cancels from the eigenvalue part, and $s$ does not contribute by assumption (is this correct?!).
Here we assume $\eta \ge 1$, or in another set of coordinates
\begin{align}
 \eta = \frac{1}{2} \left( w + \frac{1}{w} \right), \qquad
 0 < w < 1.
\end{align}
In these coordinates the block reads
\begin{align}
 f(w) = c_1 w^{2r} + c_2,
\end{align}
where the constants $c_i$ should be fixed by consistency with the OPE.



\subsection{Line defect in \texorpdfstring{$3d$}{3d}}

We can obtain a subalgebra restricting to
\begin{align}
 D, \quad
 P_3, \quad
 K_3, \quad
 -iM_{12} + R, \quad
 Q^+_1, Q^-_1 \quad
 S^{1+}, S^{1-}.
\end{align}
This can be mapped to the $\Nm = 2$ algebra in $1d$ (the left-moving part of the $\Nm = (2,0)$ algebra in $2d$).
It is natural that $M_{12}$ appears with an extra factor of $i$ compared to $R$, since in our conventions $M_{12}$ is anti-hermitian while $R$ is hermitian.
The defect Casimir now changes a bit
\begin{align}
 C_{\text{def}} = 
    D^2
  - \frac12 \{ P_{3}, K^{3} \}
  - (-iM_{12} + R)^2
  + \frac12 [ S^{1+}, Q^-_1]
  + \frac12 [ S^{1-}, Q^+_1].
\end{align}
Acting with it on an operator with quantum numbers $\Delta, \ell$, where $\ell$ is the eigenvalue of $-i M_{12} + R$ gives the eigenvalue:
\begin{align}
 \lambda_C 
 = \Delta^2
 - \ell^2.
\end{align}

\subsubsection{Defect channel}

The new contribution from susy is
\begin{align}
\begin{split}
 C_{\text{def, ferm}} 
 & = 
  - (-i M_{12} + R)^2
  + \frac12 [ S^{1+}, Q^-_1]
  + \frac12 [ S^{1-}, Q^+_1] \\
 & =
  - (-iM_{12} + R)^2
  + (-iM_{12} + R)
  - Q^-_1 S^{1+}
  + S^{1-} Q^+_1.
\end{split}
\end{align}
Acting at point 1 only the contribution from $M$ and $R$ survives and $S$ and $Q$ drop:
\begin{align}
\begin{split}
 [C_{\text{ferm}}, \phi_1(x_1) ] |0\rangle
 & = \left(   
      M_{12}^2 
    + i (2\Delta_\phi -1) M_{12}
    - \Delta_\phi(\Delta_\phi -1) ) 
 \right) \phi_1(x_1) |0\rangle.
\end{split}
\end{align}
The action of the generators is
\begin{align}
 i M_{12}   & \to \pm i \sqrt{1-\eta^2}, \\
 (M_{12})^2 & \to - C_{\text{def},2}
\end{align}
Comments:
\begin{itemize}
 \item In Lorentzian signature we could have $\eta \ge 1$, which eats the factor $i$ (what about Euclidean signature?!?).
 \item The sign of the $(M_{12})^2$ term also looks off, I would expect it would be $(M_{12})^2 = +C_{\text{def,2}}$.
\end{itemize}

In any case, the Casimir equations to be solved are
\begin{align}
\begin{split}
& (\chi^2 - 4) f^{(2,0)}(\chi, \eta) 
  + 2 \chi f^{(1,0)}(\chi, \eta) \\
& - (\eta^2 - 1) f^{(0,2)}(\chi, \eta)
  - \eta f^{(0,1)}(\chi, \eta)
  - (2\Delta_\phi - 1) \sqrt{\eta^2 -1 } f^{(0,1)}(\chi, \eta) \\
& - \left( \Delta_\phi(\Delta_\phi - 1)
  + \mathcal{C}_2 \right) f(\chi, \eta) = 0
\end{split}
\end{align}
Comments:
\begin{itemize}
 \item A superspace calculation gives no nilpotent invariants and the same Casimir equation.
 \item The two sets of equations are decoupled so solutions are products of a function of $\chi$ and a function of $\eta$. The one for $\chi$
 \begin{align}
  (\chi^2 - 4) f''(\chi) + 2 \chi f'(\chi) 
  - \left(\Delta+\frac{1}{2}\right) \left(\Delta-\frac{1}{2}\right) f(\chi) = 0,
 \end{align}
 can be easily solved into a shifted block: $f_{\Delta+1/2}(\chi)$.
 \item It would be nice to find whether the rest (or something similar up to some signs) has a solution in terms of a sum of non-supersymmetric blocks.
\end{itemize}


% Now comes the puzzle: in our conventions the action of $M_{\mu\nu}$ on $\phi$ is a real differential operator $2 x_{[\nu}\partial_{\mu]}$, so this contribution to the Casimir equation is complex.
% It might look like our conventions are inconsistent, but it's not clear how to fix it.
% In particular, if we send $M \to i M$, then the defect subalgebra contains $M_{12} + R$ but the differential operator for $M$ becomes $2 i x_{[\nu}\partial_{\mu]}$.
% Translating operators as $e^{ixP} \Om(0)$ instead of our conventions also does not make a difference.

\section{Superspace calculation \texorpdfstring{$3d$}{3d}}

We translate as
\begin{align}
 \Om(z) = e^{x^\mu P_\mu + \theta^{\a-} Q^+_\a + \theta^{\ad+} Q^-_\ad} \Om(0),
\end{align}
where the adjoint action is implicit. Thus
\begin{align}
 [P_\mu, \Om(z)] & = \partial_\mu \Om(z), \\
 [Q^+_\a, \Om(z)] & = \left( 
     \frac{\partial}{\partial \theta^{\a-}} 
   - \frac{1}{2} \Sigma^\mu_{\a\ad} \theta^{+\ad} \partial_\mu
 \right) \Om(z), \\
 [Q^-_\ad, \Om(z)] & = \left( 
     \frac{\partial}{\partial \theta^{\ad+}} 
   - \frac{1}{2} \Sigma^\mu_{\a\ad} \theta^{-\a} \partial_\mu
 \right) \Om(z), \\
 [D, \phi(x)] & =  \left(
      x^\mu \partial_\mu 
    + \theta^{\ad+} \frac{\partial}{\partial \theta^{\ad+}} 
    + \theta^{\a-}  \frac{\partial}{\partial \theta^{\a-}} 
    + \Delta
  \right) \phi(x), \\
 \ldots
\end{align}
The covariant derivatives are
\begin{align}
   D^+_\a 
 & = \frac{\partial}{\partial \theta^{\a-}} 
   + \frac{1}{2} \Sigma^\mu_{\a\ad} \theta^{+\ad} \partial_\mu, \\
   D^-_\ad
 & = \frac{\partial}{\partial \theta^{\ad+}} 
   + \frac{1}{2} \Sigma^\mu_{\a\ad} \theta^{-\a} \partial_\mu,
\end{align}
The chiral and antichiral coordinates
\begin{align}
 y^\mu = x^\mu - \frac{1}{2} \theta^- \Sigma^\mu \theta^+, \qquad
 D^+_\a y^\mu = 0, \quad
 D^+_\a \theta^+ = 0, \\
 \tilde y^\mu = x^\mu + \frac{1}{2} \theta^- \Sigma^\mu \theta^+, \qquad
 D^-_\a y^\mu = 0, \quad
 D^-_\a \theta^- = 0.
\end{align}
Imposing conservation of $Q_{1}^\pm$ and $P_3$ the invariant objects are
\begin{align}
 & z_1^1 = y_1^1 -   \theta_1^{2+} \theta_2^{1-}, \quad
   z_1^2 = y_1^2 + i \theta_1^{2+} \theta_2^{1-}, \quad
   \theta_1^{2+}, \\[0.5em]
 & \tilde z_2^1 = \tilde y_2^1 +   \theta_1^{1+} \theta_2^{2-}, \quad
   \tilde z_2^2 = \tilde y_2^2 + i \theta_1^{1+} \theta_2^{2-}, \quad
   \theta_2^{2-}, \\[0.5em]
 & z_{12}^3 = y_1^3 - \tilde y_2^3 - \theta_1^{1+} \theta_2^{1-}.
\end{align}



\appendix



\section{Useful identities}

\begin{align}
 \{ A, B C \} = \{ A, B \} C - B [A, C], \\
 \{ A, [B, C] \} = - \{ B, [A, C] \}  + [ \{A, B \}, C], \\
\end{align}


\end{document}
